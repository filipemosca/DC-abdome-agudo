\begin{frame}{História da doença atual}
Mulher de 24 anos de idade, primigesta, nulípara, com 24 semanas de gestação, G0P1A0, procurou o serviço de emergência do Hospital da Restauração apresentando sintoma principal de dor abdominal nos últimos 4 dias acompanhada de distensão. A dor era insidiosa no início, de caráter agudo, localizada no quadrante inferior direito, mas aumentou progressivamente até envolver todo o abdome. O quadro estava associado a náuseas e vômitos. Além disso, a paciente refere não defecar nos últimos dois dias e dificuldade para urinar. 
\end{frame}

\begin{frame}{História da doença atual}
    \begin{itemize}
        \item Há 4 dias:
             \begin{itemize}
                 \item Apresentou sintoma principal de dor abdominal  insidiosa, de caráter agudo, localizada no quadrante inferior direito.
             \end{itemize}
        \item Associada à:
            \begin{itemize} 
                \item Distensão, náuseas, vômitos e febre.
            \end{itemize}
        \item Melhorava com:
            \begin{itemize} 
                \item Sem fatores de melhora.
            \end{itemize}
        \item Há 2 dias:
             \begin{itemize}
                 \item A dor evoluiu para difusa em todo abdome. 
             \end{itemize}
       \item Associada à:
         \begin{itemize}
             \item Ausência de movimentos intestinais.  
         \end{itemize}
    \end{itemize}
\end{frame}